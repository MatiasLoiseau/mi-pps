% ----------------------------------
% Preambulo
% ----------------------------------
\documentclass[12pt,a4paper]{article}
\usepackage[spanish]{babel}
\usepackage[T1]{fontenc}
\usepackage{textcomp}
\usepackage{lmodern}
\usepackage[utf8]{inputenc}
\usepackage{graphicx}
\usepackage{listings}

\setcounter{secnumdepth}{3} % para que ponga 1.1.1.1..
\setcounter{tocdepth}{4} % para que añadir las secciones en el índice...


\author{Matías Loiseau}
\title{Práctica Profesional Supervisada}

% ----------------------------------
% Documento
% ----------------------------------
%INCLUIR LOGO UNDAV
\begin{document}
\begin{figure}
  \centering
   \includegraphics[width=0.2\textwidth]{undav-logo}
  %\caption{Mi Figura}
  \label{fig:ejemplo}
\end{figure}
\maketitle       % genera el título
\begin{center}
Open Computación S.A 
\end{center}
\cleardoublepage

\tableofcontents % inserta el índice general
\cleardoublepage

\section{Objetivos}
El presente trabajo consiste en explicar el rol del estudiante en un entorno profesional a través de su inserción en una realidad o ambiente laboral específico. Se desarrollará la aplicación integrada de los conocimientos que ha adquirido a través de su formación académica. Este proceso debe desarrollarse en sectores productivos y/o servicios que tengan una relación directa con los contenidos de la carrera Ingeniería en Informática.

\section{Introducción}
La Práctica Profesional Supervisada (PPS) es una actividad curricular y en consecuencia obligatoria, en la que el/la estudiante realiza actividades contempladas en los alcances del título y relacionadas con el medio real de desempeño de la profesión.

En este caso el estudiante concurre al día de la fecha a la empresa \emph{Open Computación S.A.} en el área de Infraestructura cumpliendo el labor de Administrador de Sistemas Junior. La práctica es supervisada por un profesional en la misma área en la cual trabaja junto al estudiante.

\section{Open Computación S.A.}
Sus servicios comprenden la consultoría, el diseño, el desarrollo, la instalación, la puesta a punto, y el mantenimiento de sistemas informáticos de todo tipo; en particular, aquellos orientados a Internet.

\subsection{Objetivos de la empresa}
Se encargan del 100\% de las tareas de mantenimiento y servicio técnico de las PCs, notebooks, servidores e impresoras, de cientos de clientes. Sea a través de abonos de soporte técnico de sus equipos informáticos o mediante la provisión de suscripciones de reparación de sus parques instalados.

Con lo cual, fundamentalmente, la empresa es proveedora de quienes demandan la tercerización o el outsourcing del soporte técnico de sus servidores, PCs, notebooks e impresoras, con un alcance que se extiende a Ciudad Autónoma de Buenos Aires y a toda la Provincia de Buenos Aires.

Ofrece responsabilidad y cobertura total en todo lo concerniente a respaldo de recursos digitales, servicios de back-up, seguridad en redes corporativas, y protección y confidencialidad de datos.

Provee del soporte integral y el alojamiento a una vasta comunidad de sitios web a través de nuestros propios servidores alojados en datacenters de nuestra propiedad.

\subsection{Clientes}
La empresa brinda distintos servicios y los clientes pueden comprar solo los que ellos necesiten. Estos recibirán asistencia presencial o a distancia de técnicos o administradores de sistemas que brinden soporte en su infraestructura. Cada cliente puede tener un sistema distinto como por ejemplo, que tengan contratados servidores de e-mails, hosting web-services, Firewall entre otros servicios informáticos.

Dependiendo el contrato del cliente, ellos pueden solicitar el soporte comunicándose con la empresa para que le resuelvan la problemática que les haya surgido. También dependiendo de la urgencia del problema, ellos pueden optar por comunicarse o esperar a que el técnico venga en su cita diaria, semanal, quincenal, o mensual (esto deberá estar pactado en el contrato anteriormente).

\subsection{Áreas}
La empresa esta dividida en distintas áreas donde llevan a cabo labores específicos según lo indica el nombre del área.

La empresa la coordinan 3 personas, una encargada del lado Administrativo, otra encargada de Técnica y la última de Infraestructura.

No se hará profundidad en las áreas no-técnicas porque no cumple con el objetivo del trabajo ya que involucra poco y nada al estudiante.

\subsubsection{Administración y recepción}
En estas áreas no hay mucho que explicar, solo que la presidenta de la empresa es la que esta en continuo interaccióncon las 4 personas que la componen.

Recepción es la encargada de recibir la solicitud de los clientes y debe filtrar el problema para asignarlo al área correspondiente. 

\subsubsection{Técnica}
Técnica esta conformada por 12 personas y es el área mas grande. Los técnicos cumplen varias funciones y una de estas hace que esta área sea la que mayor interacción tenga con los clientes, ya que los técnicos tienen que, mediante una cierta frecuencia, ir hacia el cliente a hacer el trabajo de consultor informático. Dependiendo el servicio contratado por el cliente, el técnico designado a dicho cliente debe resolver las problemáticas que surgen.

Los técnicos se podrían clasificar como los más multi-tareas de la empresa, ya que cumplen con reparar un ordenador, impresoras, gestionar sistemas de archivos, instalar programas, e instalar conexiones y componentes de red entre otras solicitudes que les puede llegar de los clientes.

Algunos problemas de gestión no los puede resolver el técnico por si solo y deberá llevar la solicitud hacia el área de infraestructura

\subsubsection{Infraestructura}
En esta área es la que el estudiante se esta desempeñando. Infraestructura es la que tiene más contacto con la gestión de servidores y armado de sistemas de redes para los clientes. Esta área se puede sub-dividir en dos labores que son: el armado de nuevos proyectos y cumplir la función de soporte como administradores de sistemas. La componen 6 personas, incluido el estudiante.

Continuamente se piensa en generar nuevos servicios para poder vender a los clientes, sin estos proyectos, la empresa no se actualiza y puede llegar a estar obsoleta. Este labor consiste en investigar, desarrollar y probar nuevos programas y servicios.

Cada solicitud o inconveniente que llegue a infraestructura se resuelve conectándose remotamente al cliente y se aborda el problema.

\section{Entorno de trabajo}
El administrador de sistemas (SysAdmin) cuenta con un entorno de trabajo variado lleno de información para resolver los distintos problemas que llegan al área.

\subsection{GLPi}
Toda la empresa esta inter-conectada mediante un software de GLPi\footnote{Acrónimo: en francés, Gestionnaire Libre de Parc Informatique}  que es una solución libre de gestión de servicios de tecnología de la información (ITSM), un sistema de seguimiento de incidencias y de solución service desk. Este software de código abierto está editado en PHP y distribuido bajo una licencia GPL. \cite{GLPi}

Cada vez que llega una solicitud o inconveniente se genera un ticket dentro del GLPi informando el problema. Este ticket es asignado a uno o varios responsables de resolverlo. Mientras se esta ejecutando la solución al problema, los involucrados pueden escribir comentarios o el seguimiento de sus pasos dentro de este software para que los demás puedan estar informados de los cambios o estado actual del problema.

El sistema de tickets dentro del GLPi es muy importante para la comunicación de problemas entre los técnicos o administradores de sistemas.  

\subsection{WIKI}
Como se cuenta con un número grande de clientes, para almacenar toda la información de cada cliente se usa una plataforma web propia de la empresa. En ella se encuentra toda información relevante para que los técnicos o administradores puedan consultar allí la topografía o sistemas del cliente. 

Esta wiki es muy importante, pero es mucho mas importante mantenerla actualizada porque si se tiene información vieja, la mayor parte de la información se hace obsoleta. 

\subsection{Terminal}
La terminal de linux se usa para acceder al sistema operativo sin utilizar la interfaz gráfica y realizar todo tipo de tareas en modo texto. Los trabajos que se explicarán a continuación son todos en entornos Linux y la gran mayoría tienen la funcionalidad de Firewalls o Fileservers que tienen como host un Linux server (usualmente Ubuntu o Debian) y estos no poseen entorno gráfico asi que la única manera de poder acceder es mediante la terminal.

\section{Tareas}
Al estudiante se le fueron dando tareas de dificultad gradual para que empiece a conocer la empresa por dentro y como se manejan las solicitudes e inconvenientes.

A continuación se detallará algunas tareas donde el estudiante trabajó o colaboró.

\subsection{Actualizar WIKI}
Como se dijo anteriormente, la wiki necesita continua actualización de la información. El estudiante fue designado para entrar a cada sistema de servidores o firewall de los clientes y hacer un relevamiento de los servicios que tenía corriendo, el nombre de los equipos, las direcciones IP de los mismos, y otras informaciones importantes relevantes para actualizarlas en la wiki. 

\subsubsection{Cambio de nombres}
Mientras se actualizaba la wiki, se pidió cambiar los nombres de los servidores linux para que se pueda entender que funcionalidad cumple, con solo mirar el nombre. Estos son:

FS = Fileserver

FW = Firewall

MS = Mailserver

Una solicitud extra era que cuando se logueaba al sistema con el comando \textit{visudo} se le agregara una lineal al final del archivo para que no vuelva a pedir la contraseña cuando se trataba de acceder al su con el comando \textit{sudo -i} 

La línea era:
\begin{lstlisting}
tecnico ALL = NOPASSWD:ALL
\end{lstlisting}

En algunos sistemas clientes ya estaban puestos pero no en todos. Luego de esta tarea se procedía a cambiar el nombre con el comando

\begin{lstlisting}
nano /etc/hostname
\end{lstlisting}

Se le borra el contenido del archivo y se ingresa el siguiente patrón

\begin{lstlisting}
<FS/FW/MS>-<NombreDeLaEmpresa>
\end{lstlisting}

Un ejemplo sería:

\begin{lstlisting}
FS-Maffra
\end{lstlisting}

Se guarda y se reinicia el init.d con:

\begin{lstlisting}
/etc/init.d/hostname.sh start
\end{lstlisting}

Había veces que esto solo no alcanzaba entonces se procedía con un segundo método de cambio de hostname y se hacía escribiendo:

\begin{lstlisting}
hostname
hostname --fqd
hostname NEW_NAME
\end{lstlisting}

Para efectuar el cambio de nombre se tenía que salir de la sesión y volver a entrar.

\subsection{Poner en marcha un Fileserver}
Un cliente compró un servidor nuevo para cambiar el viejo que tenían. Se habré un ticket explicando las especificaciones y la primera puesta en marcha. El ticket del técnico encargado decía: 

\begin{center}
-----------------------------------------------------------------
\end{center}

Se creó una Virtual Machine en el nuevo servidor con HYPER-V. En el servidor que se esta preparando para Turismo Cabal con Ubuntu Server 18.04, el mismo llamado "FS-TC" Se conforma de 3 discos:

- Sistema operativo 80 gb 

- Datos 250 gb 

- Backup 250 gb.

Se instaló samba, ssh, clamav y mc

Formatear discos de 250 gb y crear recursos compartidos 

CARPETA RAIZ "Fileserver" contiene 2 sub-carpetas "Escaneados" y "Usuarios"

IP: *******

usuario: *******

contraseña: *******

\begin{center}
-----------------------------------------------------------------
\end{center}

Los comandos relevantes que se llevaron a cabo para esta configuración por parte del técnico fueron:

\begin{lstlisting}
apt-get update
apt-get upgrade
apt-get install mc
apt-get install samba
apt-get install ssh
apt-get install vim
reboot 
\end{lstlisting}

\subsubsection{Update y Upgrade}
Luego de cada instalación, es muy recomendable actualizar la lista de repositorios de paquetes que trae Ubuntu. Luego dicha actualización se instala con la sentencia upgrade.

\subsubsection{Midnight Commander}
MC o Midnight Commander es un controlador de archivos visual para la consola. Tiene como característica una muy buena pantalla completa de texto plano que te permite copiar, mover y borrar archivos, directorios en todos los directorios desde la raíz. También se puede buscar archivos, correr comandos en la subshell. Por último incluye un visualizador y editor de archivos.

Midnight Commander esta basado en interfaces de texto versátiles como Ncurses o S-Lang, quienes te permiten trabajar regularmente en la consola, adentro de una X terminal, sobre conexiones SSH y todos los tipos de shells remotos.\cite{MC}

\subsubsection{Samba}
Samba es el paquete estándar de programas de interoperatividad de Windows para Linux y Unix.

Samba entrega seguridad, estabilidad, y una rápida muestra de intercambio de archivos para todos los clientes que usan los protocolos SMB/CIFS, como todas las versiones de DOS y Windows, Linux y muchos otros.

Samba es un componente importante que se puede integrar en Servidores Linux/Unix y Escritorios dentro de entornos del Active Directory.

Samba es un paquete de software que entrega administración de redes, flexibilidad y libertar en términos de configuración al momento de elegir el sistema y el equipo.\cite{SAMBA}

\subsubsection{SSH}
SSH o Secure Shell protocol, es un software que permite la administración de sistemas seguros y la transferencia de archivos sobre redes inseguras. Es isado en casi todos los datacenter a lo largo del mundo coorporativo.

El protocolo SSH usa la encriptación para asegurar la conexión entre el cliente y el servidor. Todas las autentificaciones del usuario, comandos, salidas y transferencias de archivos estan encriptadas para protegerlos ante los ataques de redes. \cite{SSH}

\subsubsection{VIM}
VIM o Vi IMproved, es una versión mejorada del editor de texto vi, presente en todos los sistemas UNIX.

La principal característica tanto de Vim como de Vi consiste en que disponen de diferentes modos entre los que se alterna para realizar ciertas operaciones, lo que los diferencia de la mayoría de editores comunes, que tienen un solo modo en el que se introducen las órdenes mediante combinaciones de teclas o interfaces gráficas. \cite{VIM}

\subsubsection{Montado de discos}
En el ticket se habló que estaban enchufados los discos pero eso no significa que estén montados al servidor. Lo que se hizo fue ver las direcciones de los discos y luego crear particiones ext4. Esto es posible con los siguientes comandos:

\begin{lstlisting}
fdisk -l
mkfs.ext4 /dev/sdb 
cfdisk /dev/sdb 
mkfs.ext4 /dev/sdb1 
blkid /dev/sdb1  
mkfs.ext4 /dev/sdc1 
blkid /dev/sdc1
\end{lstlisting}

Con las particiones creadas se modifica el archivo fstab y se le agrega las direcciones de donde se van a montar los discos. Hay que tener en cuenta que esto sigue estando en desarrollo y todavía no se cargaron todos los discos por factores externos.

\begin{lstlisting}
nano /etc/fstab
\end{lstlisting}

Y se lo deja de la siguiente manera:

\begin{lstlisting}
UUID=57b764ce-b690-11e8-905b-00155d00ae01 none swap sw 
0 0
UUID=57b764cf-b690-11e8-905b-00155d00ae01 /ext4 defaults 
0 0
UUID=5cb3c2d8-b690-11e8-905b-00155d00ae01 /home ext4 
defaults 0 0
UUID=1cebd62b-690e-4707-b51b-0eab949f4947 /home/samba/ 
ext4 defaults 0 0
UUID=dc9d82b0-f22a-4f2b-a7ac-dc86f97b54b3 /backup/ ext4 
defaults 0 0
\end{lstlisting}

Luego se crean las carpetas escritas anteriormente y se reinicia con:

\begin{lstlisting}
mkdir /backup
mkdir /home/samba
mount -a
reboot
\end{lstlisting}

\subsubsection{Creación de usuarios}
Como se había dicho anteriormente, Samba gestiona los usuarios y para esto hay que crear a los usuarios que puedan ingresar a las carpetas compartidas y a estos darles una configuración determinada porque cada usuario tiene permisos distintos. Para la creación de usuarios y con sus contraseñas los comandos son los siguientes:

\begin{lstlisting}
useradd ejemplousuario
smbpasswd -a ejemplousuario
\end{lstlisting}

Luego se configuran los permisos y para estar seguros de toda la configuración se reinician los servicios y la máquina virtual: 

\begin{lstlisting}
nano /etc/samba/smb.conf
/etc/init.d/samba-ad-dc restart
service samba restart
reboot 
\end{lstlisting}

\subsubsection{Rsnapshot}
Rsnapshot es una utilidad para hacer back-ups de filesystems basada en rsync. Rsnpashot hace copias periódicas de la computadora local y también remotas por ssh. Se instala y luego se configura con: 

\begin{lstlisting}
sudo apt-get rsnapshot
nano /etc/rsnapshot.conf
\end{lstlisting}

No es el fin de este documento explicar como configurar snapshot pero en el archivo rsnapshot se indica el directorio raíz en donde se harán las copias y luego el intervalo de tiempo, para tener una idea lo siguiente es una porción del archivo:

\begin{lstlisting}
#########################################
#     BACKUP LEVELS / INTERVALS         #
# Must be unique and in ascending order #
# e.g. alpha, beta, gamma, etc.         #
#########################################

retain	daily	6
retain	weekly	4
retain	monthly	4
#retain	delta	3
\end{lstlisting}

Luego se debe configurar el archivo crontab de Linux con:

\begin{lstlisting}
nano /etc/crontab 
\end{lstlisting}

Para que ejecute las instrucciones del rsnapshot y el archivo debería quedar de la siguiente manera:

\begin{lstlisting}
# m h dom mon dow user	command
17 *	* * *	root    cd / && run-parts --report /etc/cron.hourly
25 6	* * *	root	test -x /usr/sbin/anacron || ( cd / && run-parts --report /etc/cron.daily )
47 6	* * 7	root	test -x /usr/sbin/anacron || ( cd / && run-parts --report /etc/cron.weekly )
52 6	1 * *	root	test -x /usr/sbin/anacron || ( cd / && run-parts --report /etc/cron.monthly )
00 0	* * *	root	/usr/bin/rsnapshot daily
00 2	* * 6	root	/usr/bin/rsnapshot weekly
00 4	1 * *	root	/usr/bin/rsnapshot monthy
00 22	* * *	root	/bin/backup-noc.sh
#
\end{lstlisting}

\subsection{Administrar Fileservers} %hausler
Los incidentes o solicitudes que llegan al área de infraestructura con temas de Fileservers usualmente son peticiones para agregar, modificar o eliminar permisos de las carpetas compartidas entre los usuarios. Estas solicitudes se resuelven modificando el archivo de configuraciones de samba:

\begin{lstlisting}
nano /etc/samba/smb.conf
\end{lstlisting}

En las siguientes lineas se va a mostrar una configuración de una carpeta compartida para que se pueda observar las distintas configuraciones que se pueden hacer con samba para resolver la solicitud o inconveniente.

\begin{lstlisting}
[Suc Jose C Paz-Enfermeria]
    comment = Share para la sucursal de jose c paz de enfermeria
    path = /home/samba/Sucursal-Jose-C-Paz-Enfermeria
    guest ok = no
    force user = nobody
    force group = nogroup
    valid users = user1,user2,user3
    admin user = @admin
    force create mode = 0777
    force directory mode = 0777
    create mask = 0777
    directory mask = 0777
#    directory mask = 0770
#    force create mode = 0660
#    force security mode = 0660
#    security mask = 0770
#    file mask = 0660
    writeable = yes
\end{lstlisting}

Luego de cambiar las configuraciones se debe reiniciar el servicio de samba con:

\begin{lstlisting}
/etc/init.d/samba restart
\end{lstlisting}

\subsection{Reiniciar servicios VPN}
Descripción del ticket:

\begin{center}
-----------------------------------------------------------------
\end{center}
Reiniciar servicio de Open VPN para que se puedan enlazar por ese medio las bases del sistema plenario interno con Billy
\begin{center}
-----------------------------------------------------------------
\end{center}

Estos tipos de trabajos se resuelven de una manera sencilla y las instrucciones fueron:

\begin{lstlisting}
Last login: Mon Jul 30 15:05:58 2018 from 186.22.149.146
tecnico@maffra-fw:~$ sudo -i
[sudo] password for tecnico: 
tecnico@maffra-fw:~$ su
Contrasena: 
root@maffra-fw:/home/tecnico# /etc/init.d/openvpn restart
[ ok ] Restarting openvpn (via systemctl): openvpn.service.
root@maffra-fw:/home/tecnico# exit
tecnico@maffra-fw:~$ logout
Connection to maffra.opensa.com.ar closed.
\end{lstlisting}

Se reinició exitosamente el servicio de openvpn y el técnico encargado del ticket confirmó por medio del GLPI que funcionó.

\subsection{Mail Server} 
Se informa que se calló el servicio de mails en un cliente, el mapa de conexión es:

\begin{center}
UsuariosDeLaEmpresa ←-→ Mail-Server ←-→ Firewall
\end{center}

Se entró al Mail Server y no se detectó ningún problema. Se intento reiniciar los servicios y también el servidor. Los logs solo mostraron que los mails eran rechazados pero no daban ningún dato para detectar el problema. Algunos pasos que se dieron para probar posibles errores fueron:

\begin{lstlisting}
/etc/init.d/fetchmail restart
/etc/init.d/fetchmail start 
/etc/init.d/procps restart
tail -f
tail -f /home/info/.procmail.log
tail -f /var/log/fetchmail.log
info /usr/bin/fetchmail -s
/usr/bin/fetchmail -s
tail -f /var/log/syslog
tail -n 1000 /var/log/syslog
reboot
vim /home/info/.fetchmailrc 
kill 3461
tail -n 1000 /var/log/syslog
fetchmail -s
tail -f /var/log/syslog
cat /home/info/.fetchmailrc 
\end{lstlisting}

Luego de todos estos intentos fallidos se entro al Firewall y con el iptables se pudo observar que la aplicación fail2ban estaba bloqueando la IP del servidor (se desconoce la causa). La solución fue que se configuro esa aplicación modificando un archivo y agregando el comando

\begin{lstlisting}
ignoreip="ip"
\end{lstlisting}

\subsection{Montar Back-Up externo en un Fileserver} 
Se pidió que se haga un back-up externo al Fileserver de un cliente. Ellos tienen un NAS\footnote{Network Attached Storage (NAS), son dispositivos de almacenamiento a los que se accede desde los equipos a través de protocolos de red. Estos comparten sus unidades por red.}  pero estaba mal montado, cuando se trataba de ejecutar el comando \textit{mount -a} devolvía un error. Para realizar esta solicitud se debe montar una carpeta del NAS dentro de una carpeta del Fileserver.

Existe un script que hace el back-up en una carpeta \textit{/backups-externo/samba/} que es:

\begin{lstlisting}
rsync -e ssh -vauh --recursive --ignore-errors --exclude 
"transito/" /home/samba/ /backups-externo/samba/
\end{lstlisting}

Este script se ejecuta en el crontab, anteriormente explicado.

No todas las carpetas del NAS están compartidas, para saber que carpeta del NAS hay que montar, dentro de su terminal se tira el comando para listar todas las carpetas compartidas.

\begin{lstlisting}
smbclient -L 
\end{lstlisting}
 
Cuando ya se tiene el nombre de la carpeta del NAS se procede a montarla por el protocolo CIFS y para que se monte automáticamente. Esto es configurado desde el archivo \textit{/etc/fstab}. Al final de dicho archivo se colocó la siguiente linea:

\begin{lstlisting}
//"""IP"""/Bk-Server /backups-externo/samba	cifs	
username=user,password=****** 0  0 
\end{lstlisting}

Luego de configurarlo se vuelve a tratar de montar con el comando:

\begin{lstlisting}
mount -a
\end{lstlisting}

Si no retorna ningún error significa que se montó exitosamente, y luego se ejecuta el script previamente creado para ejecutar el back-up

\begin{lstlisting}
/bin/backup-externo.sh
\end{lstlisting}

\section{Asimilación con la carrera}
La práctica profesional escogida por el estudiante no abarca la totalidad de los objetivos de la orientación de Sistemas Distribuidos, es decir, solo pocas materias hacen un enfoque al área de sistemas como Sistemas Operativos y materias de la orientación Sistemas de información y gestión empresarial. Sin embargo a lo largo de la carrera el estudiante fue llevado a prueba en el uso de herramientas de software libre y elaboró muchas actividades de forma prácticas, es decir, no solo se dio un enfoque teórico sino que todos los conocimientos adquiridos eran probados y analizados. Esto le dio una confianza y conocimiento extra porque ya había sido entrenado en el uso de las herramientas de trabajo.

La elección del estudiante en seguir la orientación de Sistemas Distribuidos le aporto tener una base de redes y arquitecturas que eran necesarias para llevar a cabo la función como Administrador de Sistemas Junior, pero hubiera sido mas adecuada la orientación de Sistemas de información y gestión empresarial, sin embargo, en las materias Redes de Computadoras, Tecnología y Servicios de Redes, Protocolos y Arquitectura de Ruteo y Programación Distribuida 1 y 2 le aportó una base muy grande sobre las interconexiones de los sistemas con los que trabajó. Las materias específicas del área de Sistemas Distribuidos fueron un factor clave para comprender de antemano donde se alojaban y como se comunicaban los sistemas de donde se trabajaba.

\section{Conclusiones}
El estudiante pudo desplegar sus conocimientos y aptitudes con total normalidad. Se pudo defender a cada problema que le surgía, aunque si bien al principio no conocía del todo el entorno donde trabajaba, se adaptó de manera rápida y eficazmente. Pudo abordar los problemas con un enfoque de ingeniería típica. También operó y mantuvo aplicaciones, equipos y sistemas de procesamiento de la información. 

Cabe destacar que el ambiente de trabajo fue ameno y solidario. Esto favoreció la salud y estado mental del estudiante. 

Para finalizar, la pasantía del estudiante en esta empresa esta por terminar pero le ofrecieron la renovación por medio de un contrato y el accedió. Esto es la prueba que el trabajo que realizó el estudiante fue todo un éxito.

\cleardoublepage
\begin{thebibliography}{9}
\bibitem{GLPi} 
Colaboradores de Wikipedia. GLPi [en línea]. Wikipedia, La enciclopedia libre, 2018 [consultado: 10 de octubre del 2018]. Disponible en \textit{<https://es.wikipedia.org/wiki/GLPi>}

\bibitem{MC}
Desarrolladores de Midnight Commander. Midnight Commander, [consultado: 11 de octubre del 2018]. Disponible en \textit{<https://midnight-commander.org/>}  

\bibitem{SAMBA}
Desarrolladores de Samba. Samba, [consultado: 11 de octubre del 2018]. Disponible en \textit{<https://www.samba.org/samba/>}  

\bibitem{SSH}
Tatu Ylonen. SSH (Secure Shell), [consultado: 11 de octubre del 2018]. Disponible en \textit{<https://www.ssh.com/ssh/>}  

\bibitem{VIM} 
Colaboradores de Wikipedia. Vim [en línea]. Wikipedia, La enciclopedia libre, 2018 [consultado: 11 de octubre del 2018]. Disponible en \textit{<https://es.wikipedia.org/wiki/Vim>}



\end{thebibliography}



\end{document}

