% ----------------------------------
% Preambulo
% ----------------------------------
\documentclass[12pt,a4paper]{article}
\usepackage[spanish]{babel}
\usepackage[T1]{fontenc}
\usepackage{textcomp}
\usepackage{lmodern}
\usepackage[utf8]{inputenc}
\usepackage{graphicx}
\usepackage{listings}

\setcounter{secnumdepth}{3} % para que ponga 1.1.1.1..
\setcounter{tocdepth}{4} % para que añadir las secciones en el índice...


\author{Matías Loiseau}
\title{Práctica Profesional Supervisada}

% ----------------------------------
% Documento
% ----------------------------------
%INCLUIR LOGO UNDAV
\begin{document}
\begin{figure}
  \centering
   \includegraphics[width=0.2\textwidth]{undav-logo}
  %\caption{Mi Figura}
  \label{fig:ejemplo}
\end{figure}
\maketitle       % genera el título
\begin{center}
Open Computación S.A 
\end{center}
\cleardoublepage

\tableofcontents % inserta el índice general
\cleardoublepage

\section{Objetivos}
El presente trabajo consiste en explicar el rol del estudiante en un entorno profesional a través de su inserción en una realidad o ambiente laboral específico. Se desarrollará la aplicación integrada de los conocimientos que ha adquirido a través de su formación académica. Este proceso debe desarrollarse en sectores productivos y/o servicios que tengan una relación directa con los contenidos de la carrera Ingeniería en Informática.

\section{Introducción}
La Práctica Profesional Supervisada (PPS) es una actividad curricular y en consecuencia obligatoria, en la que el/la estudiante realiza actividades contempladas en los alcances del título y relacionadas con el medio real de desempeño de la profesión.

En este caso el estudiante concurre al día de la fecha a la empresa \emph{Open Computación S.A.} en el área de Infraestructura cumpliendo el labor de Administrador de Sistemas Junior. La práctica es supervisada por un profesional en la mísma área en la cual trabaja junto al estudiante.

\section{Open Computación S.A.}
Sus servicios comprenden la consultoría, el diseño, el desarrollo, la instalación, la puesta a punto, y el mantenimiento de sistemas informáticos de todo tipo; en particular, aquellos orientados a Internet.

\subsection{Objetivos de la empresa}
Se encargan del 100\% de las tareas de mantenimiento y servicio técnico de las PCs, notebooks, servidores e impresoras, de cientos de clientes. Sea a través de abonos de soporte técnico de sus equipos informáticos o mediante la provisión de suscripciones de reparación de sus parques instalados.

Con lo cual, fundamentalmente, la empresa es proveedora de quienes demandan la tercerización o el outsourcing del soporte técnico de sus servidores, PCs, notebooks e impresoras, con un alcance que se extiende a Ciudad Autónoma de Buenos Aires y a toda la Provincia de Buenos Aires.

Ofrece responsabilidad y cobertura total en todo lo concerniente a respaldo de recursos digitales, servicios de back-up, seguridad en redes corporativas, y protección y confidencialidad de datos.

Provee del soporte integral y el alojamiento a una vasta comunidad de sitios web a través de nuestros propios servidores alojados en datacenters de nuestra propiedad.

\subsection{Clientes}
La empresa brinda distintos servicios y los clientes pueden comprar solo los que ellos necesiten. Los clientes recibirán asistencia presencial o a distancia de técnicos o administradores de sistemas que brinden soporte en su infraestructura. Cada cliente puede tener un sistema distinto como por ejemplo, que tengan contratados servidores de e-mails, hosting web-services, firewall entre otros servicios informáticos.

Dependiendo el contrato del cliente, ellos pueden solicitar el soporte comunicandose con la empresa para que le resulvan la problemática que les haya surgido. También dependiendo la urgencia del problema, ellos pueden optar por comunicarse o esperar a que el técnico venga en su cita diaria, semanal, quincenal, o mensual (esto deberá estar pactado en el contrato anteriormente).

\subsection{Áreas}
La empresa esta dividida en distintas áreas donde llevan a cabo labores específicos según lo indica el nombre del área.

La empresa la coordinan 3 personas, una encargada del lado Administrativo, otra encargada de Técnica y la última de Infraestructura.

No se hará profundidad en las áreas no-técnicas porque no cumple con el ojetivo del trabajo ya que involucra poco y nada al estudiante.

\subsubsection{Administración, recepción y contaduría}
En estas aŕeas no hay mucho que explicar, solo que la presidenta de la empresa es la que esta en continuo interacción entre todas ellas y la componen al rededor de \textbf{xxx} personas.

Recepción es la encargada de recibir la solicitud de los clientes y debe filtrar el problema para asignarlo al área correspondiente. 

\subsubsection{Tecnica}
Técnica esta conformada por aproximadamente un grupo de \textbf{xxx} personas. Los técnicos cumplen varias funciones y una de estas hace que esta área sea la que mayor interacción tenga con los clientes, ya que los técnicos tienen que, mediante una cierta frecuencia, ir hacia el cliente a hacer el trabajo de consultor informático. Dependiendo el servicio contratado por el cliente, el técnico designado a dicho cliente debe resolver las problemáticas que surgen.

Los técnicos se podrían clasificar como los más multi-tareas de la empresa, ya que cumplen las tareas de reparar un ordenador, impresora, gestionar sistemas de archivos, instalar programas, e instalar conexiones y componentes de red entre otras solicitudes que les puede llegar de los clientes.

Algunos problemas de gestión no los puede resolver el técnico por si solo y deberá llevar la solicitud hacia el área de infraestructura

\subsubsection{Infraestructura}
En esta área es la que el estudiante se esta desempeñando. Infraestructura es la que tiene más contacto con la gestión de servidores y armado de sistemas de redes para los clientes. Esta área se puebe sub dividir en dos labores que son: el armado de nuevos proyectos y cumplir la función de soporte como administradores de sistemas. 

Continuamente se piensa en generar nuevos servicios para poder vender a los clientes, sin estos proyectos, la empresa no se actualiza y puede llegar a estar obsoleta. Este labor consiste en investigar, desarrollar y probar nuevos programas y servicios.

Cada solicitud o inconveniente que llegue a infraestructura se resuelve conectándose remotamente al cliente y se aborda el problema.

\section{Entorno de trabajo}
El administrador de sistemas (SysAdmin) cuenta con un entorno de trabajo variado lleno de información para resolver los distintos problemas que llegan al área.

\subsection{GLPi}
Toda la empresa esta interconectada mediante un software de GLPi (acrónimo: en francés, Gestionnaire Libre de Parc Informatique) que es una solución libre de gestión de servicios de tecnología de la información (ITSM), un sistema de seguimiento de incidencias y de solución service desk. Este software de código abierto está editado en PHP y distribuido bajo una licencia GPL. \cite{GLPi}

Cada vez que llega una solicitud o inconveniente se genera un ticket dentro del GLPi informando el problema. Este ticket es asignado a uno o varios responsables de resolver dicho ticket. Mientras se esta ejecutando la solución al problema, los involucrados pueden escribir comentarios o el seguimiento de sus pasos dentro de este software para que los demas puedan estar informados de los cambios o estado actual del problema.

El sistema de tickets dentro del GLPi es muy importante para la comunicación de problemas entre los técnicos o administradores de sistemas.  

\subsection{WIKI}
Como se cuenta con un número grande de clientes, para almacenar toda la informacíon de cada cliente se usa una plataforma web propia de la empresa. En ella se encuentra toda información relevante para que los técnicos o administradores puedan consultar allí la topografía o sistemas del cliente. 

Esta wiki es muy importante, pero es mucho mas importante mantenerla actualizada porque si se tiene información vieja, la mayor parte de la información se hace obsoleta. 

\subsection{programa3}
...

\section{Tareas}
Al estudiante se le fueron diendo tareas de dificultad gradual para que empiece a conocer la empresa por dondetro y como se manejan las solicitudes e inconvenientes.

A continuación se detallará algunas tareas donde el estudiante trabajó o colaboró.

\subsection{Actualizar WIKI}
Como se dijo anteriormente, la wiki necesita continua actualización de la información. El estudiante fue designado para entrar a cada sistema de servidores o firewall de los clientes y hacer un relevamiento de los servicios que tenía corriendo, el nombre de los equipos, las direcciones IP de los mismos, y otras informaciones importantes relevantes para actualizarlas en la wiki. 

\subsubsection{Cambio de nombres}
Mientras se actualizaba la wiki, se pidió cambiar los nombres de los servidores linux para que se pueda entender que funcionalidad cumple, con solo mirar el nombre. Estos son:

FS = Fileserver

FW = Firewall

MS = Mailserver

Cuando se logueaba al sistema se pedía que con el comando \textit{visudo} se le agregara una lineal al final del archivo para que no vuelva a pedir la contraseña con el comando \textit{sudo -i} 

La línea era:
\begin{lstlisting}
tecnico ALL = NOPASSWD:ALL
\end{lstlisting}


Con eso ya cambiado (algunas ya lo tenían) se procede a cambiar el nombre de la siguiente manera: se cambia el hostname con:

nano /etc/hostname

y se le borra el contenido y se pone:

<FS/FW/MS>-<NombreDeLaEmpresa>

se guarda y se reinicia el init.d:
/etc/init.d/hostname.sh start

para asegurarse se hace:

%hostname
%hostname --fqd
%hostname NEW_NAME

y ahora reinicio el ocs con :
ocsinventory-agent 

(si no cambio el nombre se sale y entra de la sesion y queda OK)

\subsection{Poner en marcha un Fileserver}
...

\subsection{Administrar Fileservers} %hausler
...

\subsection{Firewall Maffra}
...

\subsection{Firewall Amarante} %amarante problem
...

\subsection{Back-Ups} 
...

\subsubsection{rsnapshot} %explicar un poco que es rsnapsot y como lo usamos
...

\subsubsection{Montar Back-Up externo en un Fileserver}
...

\section{Asimilación con la carrera}
...

\section{Conclusiones}
...

\begin{thebibliography}{9}
\bibitem{GLPi} 
Colaboradores de Wikipedia. GLPi [en línea]. Wikipedia, La enciclopedia libre, 2018 [consulta: 10 de octubre del 2018]. Disponible en \textit{<https://es.wikipedia.org/wiki/GLPi>}
\end{thebibliography}

\end{document}

