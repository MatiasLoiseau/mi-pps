% ----------------------------------
% Preambulo
% ----------------------------------
\documentclass[12pt,a4paper]{article}
\usepackage[spanish]{babel}
\usepackage[T1]{fontenc}
\usepackage{textcomp}
\usepackage{lmodern}
\usepackage[utf8]{inputenc}
\usepackage{graphicx}

\setcounter{secnumdepth}{3} % para que ponga 1.1.1.1..
\setcounter{tocdepth}{4} % para que añadir las secciones en el índice...


\author{Matías Loiseau}
\title{Práctica Profesional Supervisada}

% ----------------------------------
% Documento
% ----------------------------------
%INCLUIR LOGO UNDAV
\begin{document}
\begin{figure}
  \centering
   \includegraphics[width=0.2\textwidth]{undav-logo}
  %\caption{Mi Figura}
  \label{fig:ejemplo}
\end{figure}
\maketitle       % genera el título
\begin{center}
Open Computación S.A 
\end{center}
\cleardoublepage

\tableofcontents % inserta el índice general
\cleardoublepage

\section{Objetivos}
El presente trabajo consiste en explicar el rol del estudiante en un entorno profesional a través de su inserción en una realidad o ambiente laboral específico. Se desarrollará la aplicación integrada de los conocimientos que ha adquirido a través de su formación académica. Este proceso debe desarrollarse en sectores productivos y/o servicios que tengan una relación directa con los contenidos de la carrera Ingeniería en Informática.

\section{Introducción}
La Práctica Profesional Supervisada (PPS) es una actividad curricular y en consecuencia obligatoria, en la que el/la estudiante realiza actividades contempladas en los alcances del título y relacionadas con el medio real de desempeño de la profesión.

En este caso el estudiante concurre al día de la fecha a la empresa \emph{Open Computación S.A.} en el área de Infraestructura cumpliendo el labor de Administrador de Sistemas Junior. La práctica es supervisada por un profesional en la mísma área en la cual trabaja junto al estudiante.

\section{Open Computación S.A.}
Sus servicios comprenden la consultoría, el diseño, el desarrollo, la instalación, la puesta a punto, y el mantenimiento de sistemas informáticos de todo tipo; en particular, aquellos orientados a Internet.

\subsection{Objetivos de la empresa}
Se encargan del 100\% de las tareas de mantenimiento y servicio técnico de las PCs, notebooks, servidores e impresoras, de cientos de clientes. Sea a través de abonos de soporte técnico de sus equipos informáticos o mediante la provisión de suscripciones de reparación de sus parques instalados.

Con lo cual, fundamentalmente, la empresa es proveedora de quienes demandan la tercerización o el outsourcing del soporte técnico de sus servidores, PCs, notebooks e impresoras, con un alcance que se extiende a Ciudad Autónoma de Buenos Aires y a toda la Provincia de Buenos Aires.

Ofrece responsabilidad y cobertura total en todo lo concerniente a respaldo de recursos digitales, servicios de back-up, seguridad en redes corporativas, y protección y confidencialidad de datos.

Provee del soporte integral y el alojamiento a una vasta comunidad de sitios web a través de nuestros propios servidores alojados en datacenters de nuestra propiedad.

\subsection{Áreas}
La empresa esta dividida en distintas áreas donde llevan a cabo labores específicos según lo indica el nombre del área.

La empresa la coordinan 3 personas, una encargada del lado Administrativo, otra encargada de Técnica y la última de Infraestructura.

No se hará profundidad en las áreas no-técnicas porque no cumple con el ojetivo del trabajo ya que involucra poco y nada al estudiante.

\subsubsection{Administración, recepción y contaduría}
En estas aŕeas no hay mucho que explicar, solo que la presidenta de la empresa es la que esta en continuo interacción entre todas ellas y la componen al rededor de \textbf{xxx} personas.

\subsubsection{Tecnica}
...

\subsubsection{Infraestructura}
...

\section{Entorno de trabajo}
...

\subsection{GLPI}
...

\subsection{WIKI}
...

\subsection{programa3}
...

\section{Tareas}
...

\subsection{Actualizar WIKI}
...

\subsection{Poner en marcha un Fileserver}
...

\subsection{Administrar Fileservers}
...

\subsection{Firewall Maffra}
...

\subsection{Back-Ups}
...

\subsubsection{rsnapshot}
... explicame poco

\subsubsection{Montar Back-Up externo en un Fileserver}
...

\subsection{Cambio de nombres}
...

\section{Conclusiones}
...

\end{document}

